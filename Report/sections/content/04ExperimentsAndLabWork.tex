\chapter{Experiments and Lab Work}\label{ch:experiment}
To relate the theory behind controlling pumps to the real world,
we conducted different experiments,
to gain knowledge about the system,
test controller implementations and gather data needed for controller tuning.


\section{Data Acquisition}\label{sec:data_gathering}
All data was collected with the help of custom made 
ML scripts and SLRT models.
The execution happened on the xPC Target OS.
Some specifics of these files will be explained in this chapter,
while the complete files can be found on the GitHub repository \cite{GitHub}.

The system is built as described in Chapter \ref{ch:physsetup},
with the controlling PC running xPC Target,
a real-time OS for use with SLRT.

Tu use SLRT on xPC targets, a SL model has to be made on the host PC,
compiled, transferred over Ethernet and executed on the target.
After the execution is finished on the target,
the .dat files containing the recorded data have to be transferred back to the host,
where further analysis can be done.

To automate as much of this process as possible,
we created a ML script that updates and compiles the SL model, transfers it to the target,
starts the execution and copies the generated .dat files to the host.

All data is recorded on the target PC from


The script programmed for the system test was done in Matlab. 
A model of the system is made using Simulink blocks. 
Simulink Real-time and xPC Target was used to run the model in real-time on 
the pump systems dedicated PC. 
\todo[color=05ExperimentsAndLabWork]{Is this ok here? Do we want a seperate section for this anyway?}




% Explain how we carried out the performance test
xPC Target allows you to add input/output blocks to your model, and then use the host 
PC and a C compiler to create executable code. The executable code is download 
from the host PC to the target PC running the xPC Target real-time kernel. 
After downloading the executable code, you can run and test your target 
application in real-time. \todo[color=05ExperimentsAndLabWork]{is it necessary to describe this and data acquisition?}

\missingfigure[figwidth=0.5\textwidth]{Placeholder figure}

Remove this nonsense:
A performance curve is plotted to indicate the variation of pump differential head against volumetric flow (gpm) of a liquid at an indicated rotational speed or velocity, while consuming a specific quantity of horsepower (BHP). The performance curve is actually four curves relating with each other on a common graph. These four curves are:

\section{System Test}\label{sec:system_test} 
To obtain information about how the system reacts under different conditions,
a test was carried out with some example conditions.
From those conditions, the goal is to extrapolate an equation for the system,
that can predict or estimate \todo[color=05ExperimentsAndLabWork]{choose predict or estimate} the systems reaction at different conditions.
To obtain the data needed,
a single pump is run at different speeds,
while flow resistance is varied by a choke valve,
resulting in corresponding values of flow, pressure, and energy consumption
measured by the sensors.
The measured data was stored for further analysis such as creating pump curves.

To get an overview over the whole operating range of the setup,
a very broad test was conducted.
We wanted to find the correlations between pump speed, backpressure, flow and power consumption.
To get reliable results, we chose to only change one variable at a time.
Since the value to be changed by our controller was expected to be the pump speed $\omega P_{1,2,3}$,
we decided to fix the backpressure by fixing the valve position,
and stepwise change $\omega P_{1,2,3}$.
Since the three pumps in the setup are expected to be identical,
the test was only run with one of the pumps.
We chose to use pump 2 at random.

\todo[color=05ExperimentsAndLabWork]{more detail later}

\section{Results}\label{sec:results}
Explain this better:\newline
- The pump speed is gradually turned up from 0 to 100\% in 10\% intervals. \newline
- For each pump speed, the valve openningvalve settling time, steps of 10 seconds\newline
- Select valve opening(backpressure) by for loop\newline

\missingfigure[figwidth=\textwidth]{Show curves from early test runs}