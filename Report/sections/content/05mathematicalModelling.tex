\chapter{Mathematical Modelling}\label{ch:mathmodel}

\todo[color=04mathematicalModelling,inline]{break this chapter into static and dynamic modelling}
Most physical systems can be modelled statically or dynamically.
Depending on the application of the model,
and therefore on the end goal of the controller,
either option has some benefits and flaws.

Since our focus shifted during the project, we started out developing a static model,
describing the steady state response of the system.
Later on we decided to move towards a dynamic model, for easier PID-controller development.

Both modelling processes are described in this chapter,
with a small section in the end explaining the key differences.
\todo[color=04mathematicalModelling]{write a small conclusion at the end of this chapter}
\section{Static Modelling}\label{sec:statmod}

The static model explains the behaviour of the system at steady-state,
i.e. when the output is settled after a step input.
Static models are therefore also referred to as steady-state models.
Both expressions are used interchangeably throughout this report.

Typical static models for pump systems are pump curves and system curves.
\todo[color=04mathematicalModelling,inline]{explain differences between pump curves and system curves}
\todo[color=04mathematicalModelling,inline]{make sure sections and subsection makes sense}

We used the data gathered through experiments to develop a model that would fit all our data, within a certain degree of error, 
and that we would be able to further test.

We decided to use grey-box modelling with polynomial fitting.
We consider our approach as grey-box rather than black-box,
because we chose the degrees of the polynomials based on known and tested physical relations.

A very similar modelling process was already successfully used by Pedersen and Yang \cite{YangMultiPump2008} on the same setup.
The choice of degree for the polynomials was determined by the affinity laws \cite{Volk2014}
and supported by very good fit to the data.

MLs Curve Fitting Toolbox (\textit{cftool}) \cite{cftool}
was used in order to find the polynomial coefficients most accurately describing the system.

\subsubsection{Single Variable Speed Pump Model}
Equation \ref{eq:pump_model} determines the head and the power, given the flow. 
Although the formula does not directly relate to the pump speed, it indirectly relates to it, due to the fact that only one possible 
pump speed $\omega$ exists for a given flow.
\todo[color=04mathematicalModelling]{maybe say that one pump speed corresponds to a certain flow if everything else is constant}
\todo[color=04mathematicalModelling]{quote zhenyu paper again}

\begin{equation}
	\begin{aligned}
	H = \bar{a_{0}} \cdot Q^2 + \bar{a_{1}} \cdot Q + \bar{a_{2}} \\
	P = \bar{b_{0}} \cdot Q^3 + \bar{b_{1}} \cdot Q^2 + \bar{b_{2}} \cdot Q + \bar{b_{3}}
	\end{aligned}
	\label{eq:pump_model}
\end{equation}

The coefficients are determined by the pump characteristics and can be experimentally identified.

Taking variable speed into account, the equations \ref{eq:pump_model} will depend on the motor speed. 
The pump Affinity Laws state:

\begin{align}
	\left(\frac{\omega_1}{\omega_2}\right)^1 = \frac{Q_1}{Q_2} && 
	\left(\frac{\omega_1}{\omega_2}\right)^2 = \frac{H_1}{H_2} &&
	\left(\frac{\omega_1}{\omega_2}\right)^3 = \frac{P_1}{P_2}	
\end{align}

Assuming the pump model parameters ($\bar{a_{0}}$, $\bar{a_{1}}$, $\bar{a_{2}}$) and ($\bar{b_{0}}$, $\bar{b_{1}}$, $\bar{b_{2}}$, $\bar{b_{3}}$) 
described in equation \ref{eq:pump_model} are obtained at a certain speed $\bar{\omega_{0}}$, 
this results in the following equation describing the pump at any speed $\omega$.
\begin{equation}
	\begin{aligned}
	H(\omega) = a_0 \cdot \omega^2 + a_1 \cdot \omega \cdot Q(\omega) + a_2 \cdot Q(\omega)^2 \\
	P(\omega) = b_0 \cdot \omega^3 + b_1 \cdot \omega^2 \cdot Q(\omega) + b_2 \cdot \omega \cdot Q(\omega)^2 + b_3 \cdot Q(\omega)^3
	\end{aligned}
\end{equation}

The coefficients can be determined as follows:

\begin{align*}
	a_0 = \frac{\bar{a_0}}{\bar{\omega_0^2}} && a_1 = \frac{\bar{a_1}}{\bar{\omega_0}} && a_2 = \bar{a_2} \\
	b_0 = \frac{\bar{b_0}}{\bar{\omega_0^3}} && b_1 = \frac{\bar{b_1}}{\bar{\omega_0^2}} && b_2 = \frac{\bar{b_2}}{\omega_0} && b_3 = \bar{b_3}
\end{align*}

Results of the model can be seen in figure and figure.
\todo[color = 04mathematicalModelling]{add figures}
\subsubsection{Multi Variable Speed Pump Model at Same Speed}
For a certain number of pumps P in parallel, with the requirement that they all run at the same speed $\omega$, the pump model is
described by the following equations:

\begin{equation}
	\begin{aligned}
	H_s(\omega) = a_0^s \cdot \omega^2 + a_1^s \cdot \omega \cdot Q_s(\omega) + a_2^s \cdot Q_s(\omega)^2 \\
	P_s(\omega) = b_0^s \cdot \omega^3 + b_1^s \cdot \omega^2 \cdot Q_s(\omega) + b_2^s \cdot \omega \cdot Q_s(\omega)^2 + b_3^s \cdot Q_s(\omega)^3
	\end{aligned}
\end{equation}

The formulas are identical, however the coefficients differ. They can be determined as follows:
\begin{align*}
	a_0^s = a_0 && a_1^s = \frac{a_1}{P} && a_2^s = \frac{a_2}{P^2} \\
	b_0^s = b_0 && b_1^s = \frac{b_1}{P} && b_2^s = \frac{b_2}{P^2} && b_3^s = \frac{b_3}{P^3}
\end{align*}
\todo[color=04mathematicalModelling]{very much from zhenyu's papers, should be able to understand and explain, otherwise, will take out and rewrite}

\newpage
The model has an adjusted coefficient of determination  $\bar{R^2}$ = 1. Such a high coefficient of determination, was achieved due to
heavy filtering of the data gathered. In addition, no, or minor disturbances were present during the gathering of the data.
\todo[color=04mathematicalModelling]{from supervisor meeting} 

The model could be more expanded, reaching a higher order polynomial. However, we have decided the model represents the pump behavior 
accurately enough. Further expansion would result in a higher degree polynomial but not much gain in terms of accuracy.

Using the model shown in Equation \ref{eq:simulated_power} we were able to replicate our data with a small error percentage. The results
are shown in the figures below.
\todo[color=04mathematicalModelling]{not sure if we need this or not so i didn't put any photos}