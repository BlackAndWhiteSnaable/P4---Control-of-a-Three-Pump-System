\chapter{Problem Description}\label{ch:probdesc}

\section{Problem Description}
We want to control a constant flow with a minimum power consumption.

This project is about different mathematical modelling approaches for a system consisting mainly of three pumps.
The modelling will be used to develop a controller governing the total flow $Q_{tot}$.
The system was already fully functionally available in our university.
No alterations on the setup were possible to complete our project,
since the system was simultaneously used by two other groups.
As controllable inputs
\todo[color=02problemDescription]{read in control book if controllable inputs is correct}
to the system were available the individual speed of each pump $\omega P_{1,2,3}$ and the valve opening of one control valve $CV_1$.

Measurable outputs of the system include
\todo[color=02problemDescription]{finish this list}
individual flow, individual differential pressure, pressure over $CV_1$, individual power consumption.
Not all of these measurements were used in our project.

A schematic representation of the system is available in Appendix X.
\todo[color=02problemDescription]{create schematic overview of the system, (\url{https://en.wikipedia.org/wiki/Piping_and_instrumentation_diagram})}


\section{Control Methods}
Based on the properties of a system two different approaches to controlling it can be applied.
Single Input Single Output (SISO) systems are often modelled in the frequency domain
as transfer functions and controlled with the help of classical control.
Here most commonly a full-output feedback loop is used because of its simplicity.
\todo[color=02problemDescription]{check if everything said here is true}
\todo[color=02problemDescription]{find relevant pages in control book}
\cite{Franklin2014}

Multiple Input Multiple Output (MIMO) systems are often modelled and controlled using state-space
modelling, where all inputs are combined into one input vector
and the plant is described as a combination of three to four matrices.
In modern control this approach is commonly used in combination with full state feedback,
where instead of the output an intermediate product (the states) are used for full-state feedback.
\todo[color=02problemDescription]{reread and rewrite}

Our system is effectively a MISO system.



\todo[color=02problemDescription]{this is obviously not done}


\section{Problem Delimitation}
\todo[color=02problemDescription]{do we need this?}
\todo[color=02problemDescription]{we need to make requirements for the controller somewhere, probably here}