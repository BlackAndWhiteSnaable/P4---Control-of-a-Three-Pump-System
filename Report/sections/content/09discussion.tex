\chapter{Discussion}\label{ch:discussion}
\todo[color=09discussion]{Discussion no text yet}

In this chapter we discuss the initially stated requirements
and compare them with the results achieved.
We will also have a look at future work opportunities,
to be done on top of this project.\\
This chapter will also explain the process of working on this project.

We initially set out three primary goals for this project:

\begin{itemize}
\item Creation of a dynamic model for one pump
\item Design of a PID controller for the flow
\item Tuning of said PID controller 
\end{itemize}
And three secondary goals:

\begin{itemize}
\item Creation of a static model for one pump
\item Creation of a static model for multiple pumps
\item Design of a controller for the flow, taking efficiency into account
\end{itemize}
Looking back at the project,
we can now say that all primary goals
and some of the secondary goals were met.
\\
We created a dynamic model for a single pump
through analysis of experimental data
and assuming a transfer function of the form 
$\frac{Y(s)}{U(s)}=\frac{A e^{s t_d}}{\tau s + 1}$.
Where all coefficients were found analysing a step response.
\todo[color=09discussion]{can we simulate with the controller?}
\\

1. This model gave a good fit to our step response data,
but performed poorly to test the actual controller.
We expect this to be an issue with the way we implemented the controller simulation in ML and SL.
\\

2. This model gave a good fit to our collected data
and was also able to predict the outcome of our proposed controller.

The design and tuning of the PID controller was primarily done on the physical setup,
instead of the simulation, because it was readily accessible
and provided good results.
This is also the reason for the dynamic model not being our first priority when it came to time management.
\\\\We initially set out to describe the whole system,
with all three pumps as a MIMO system and be able to control the total flow with minimal power consumption.
This was to be achieved by using multiple pumps and benefiting from the shifted maximum efficiency point as stated in previous research.
When we realised that we would not be able to finish that project in the given timeframe,
we chose to implement a PID control on a SISO system instead.
The knowledge gained about the system from developing the static model was helpful to determine an operating range for the PID control.
It showed that almost identical dynamics were to be expected at all points
$0\%<CV_1\leq100\%$ and $10\%\leq\omega P\leq100\%$.
It also evens the path for future work on energy efficiency,
because there already exists a reliable model of the power consumption with respect to H and Q,
which can easily be extended to a model of the efficiency.

\begin{itemize}
\item Maximum Overshoot $M_p = 0\%$
\item Steady-state error $e_{ss} \leq 1 \%$
\end{itemize}

With the manually tuned PID controller,
all requirements set in \ref{sub:req} and shown above were hit,
but this was done by making the system very slow and creating a very big delay in reaction.

With more intensive tuning and more reasonable requirements,
better coefficients for a PID controller could possibly be found.
Specifically requiring no overshoot is not expected to be reasonable in most applications of a flow controller.
It might for example be more beneficial to require the integral of the error to be very small,
to ensure steady flow on average.

\section{Future Work} \label{sec:future}
As with every project, not all work on this topic is done yet.
We therefore propose a small list of future work opportunities.
Some of the points on that list are inspired by our initial goal,
energy efficiency.

Development of a more robust dynamic model.
As stated before,
\todo[color=09discussion]{stated where?}
our dynamic model was not a perfect description of the system.
With more advanced modelling techniques and more research into this,
a better model could be found.
This would benefit most other future work.

While developing our static model,
we found that it was not accurate for the whole operating range of the system.
We believe it is possible to find a model better suited,
if more work is put into this topic.
This would also benefit most other future work
and maybe also the general understanding of pumping systems.

Research in the industry to find what requirements actually matter should be conducted,
to ensure that the next iteration of this controller would be useful.
Based on our research alone,
we cannot know which factors to prioritise
and therefore not build a beneficial controller.

Our initial goal of efficiency optimisation was not met,
due to a shift in focus.
We still think this is a worthwhile goal to work on.
Since we already put some thought into the topic,
we propose building a decisive logic to decide how many pumps to use,
which might be possible to implement as a lookup table.
A more reliable approach would of course be modelling the efficiency of 1, 2 and 3 pumps according to Q and H
and developing a MIMO controller taking those factors into account.
Developing a reliable model for the efficiency could help building a lookup table or decisive logic to switch between multiple pumps.
