\chapter*{Acronyms \& Nomenclature}
\begin{tabular*}{\textwidth}{@{\extracolsep{\fill}} l l r}
	\textbf{Symbol}	& \textbf{Definition}			& \textbf{Unit}\\
	\hline
	CL			& Closed-Loop						& \\
	DAQ 		& Data Acquisition		 			& \\
	H			& Head								& $bar$\\
	MFM 		& Magnetic Flow Meter 				& \\
	NPSH		& Net Positive Suction Head 		& $bar$\\
	OL			& Open-loop							& \\
	$P$			& Power								& $W$\\

	$\Delta P$	& Pressure Difference				& $bar$\\
	$\eta$		& Efficiency						& $\%$\\
	$\rho$		& Density							& $\frac{kg}{m^3}$\\
	$\omega$	& Rotational speed					& $\frac{rad}{s}$\\
	\hline \hline
				& 									&	\\
				& \textbf{Subscripts}				&	\\
	\hline
	$els$		& Electric							&	\\
	$hyd$		& Hydraulic							&	\\
	$ref$		& Reference							&	\\
	$tot$		& Total								&	\\
	\hline \hline
				& 									&	\\
				& \textbf{Prescripts}				&	\\
	\hline
	$\Delta$	& Change							&	\\
	\hline \hline
\end{tabular*}
\todo[color=01introduction]{We have to ctr+fuck-it for every acronym, abbreviation or nomenclature, just to make sure we actually use it}


\chapter{Introduction}\label{ch:introduction}
The human body is made of 60\% water \cite{HumanWater}.
To remain healthy,
a human needs to consume between 2,7 l and 3,7 l of water every day. \cite{DailyWater}
This amount comes from both drinks and also food.
For food to contain liquid it has to be present,
when the plants are growing,
when they are processed or added in the process of food preparation.
For drinks it has to be transported to the consumer,
in form of bottles, tanks or pipelines.

In all of those processes,
it is important to have a controllable flow,
maybe to meter the usage of water,
or to ensure the stability of a process.\\
Constant flow can also be important for different industries.
Examples range from the oil and gas sector\cite{OilFlow} over breweries\cite{BrewFlow},
dairy plants\cite{DairyFlow} to waste water treatment plants\cite{WastewaterFlow}.
In every commercial use,
it is important to keep cost low.
The lifetime cost of any system depends both on the initial cost,
but especially the running cost. \cite{LifetimeCost}
The running cost of controlling flow depends largely on the efficiency of the pumps used.\cite{LifeCycleCostEfficiency}

Prior work in the field shows, that the efficiency can be influenced by using multiple pumps,
when those are optimally scheduled. \cite{YangMultiPump2008}\\
The mentioned research only focused on controlling pressure though,
and we therefore want to investigate whether this also holds for flow control.

The focus of this report therefore lays on controlling a constant flow,
inside a working range optimized for minimal energy consumption.
\todo[color=01introduction]{Introduction done, but should be reread}
%
%
%For the water to come into the food,
%plants need to be watered regularly.
%For drinking water (or other liquids), they have to be transported.
%
%In any way, the flow of the water has to be controlled at some point in this chain,
%be it for irrigation or for bottling drinks.
%
%Even though 
%
%The focus of this project is to develop a controller for a system of pumps in such a way,
%that a constant flow is achieved with a minimal power consumption.
%
%The main focus is to understand, develop and implement modelling and control techniques.
%We will compare different techniques for both modelling and control in this report
%and conclude why specific techniques fit better to our purpose.
%
%
%Flow control is important in several situations throughout many industries.
%
%
%Controlled flow can be achieved relatively easily by controlling the shaft rotation speed $\omega$,
%but this can lead to extreme power consumption.
%\todo[inline,color=01introduction]{refer to a later chapter, something about affinity laws linear vs. cubed}
%
%According to the affinity laws%\cite{AffinityLaws}
%

\section*{Reading Guide}
\todo[color=01introduction]{what is supposed to be in here? a short explanation of what knowledge each chapter conveys?}

In Chapter \ref{ch:physsetup} the physical setup gets explained,
giving a short introduction to all components relevant for this report.
Chapter \ref{ch:mathmodel} and \ref{ch:modValPerf}
focus on the mathematical modelling of those components and their interdependencies.
Focus is mainly put on the relations between Flow, Head and Efficiency according to a given Pump Speed.
Different modelling techniques are explained and evaluated,
alongside a very brief introduction to MATLAB\textsuperscript{\textregistered{}},
Simulink\textsuperscript{\textregistered{}} Real-Time\texttrademark{} and xPC Target,
the software used for this process.
Chapter \ref{ch:experiment} explains how the experiments were conducted
and what results were achieved.
Chapter \ref{ch:controldesign} covers how the model was used to design a controller
fulfilling our requirements stated in Chapter \ref{ch:probdesc},
the implementation of which,
again using
MATLAB\textsuperscript{\textregistered{}},
Simulink\textsuperscript{\textregistered{}} Real-Time\texttrademark{}
and xPC Target gets explained in Chapter \ref{ch:cimplement}.
Chapters \ref{ch:discussion} and \ref{ch:conclusion} summarize the findings from this work,
propose possible future work and pose a conclusion to the work done.
