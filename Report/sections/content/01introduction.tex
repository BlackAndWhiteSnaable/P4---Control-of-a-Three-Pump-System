\chapter*{Acronyms \& Nomenclature}
\begin{tabular*}{\textwidth}{@{\extracolsep{\fill}} l l r}
	\textbf{Symbol}	& \textbf{Definition}			& \textbf{Unit}\\
	\hline
	CL			& Closed-Loop						& \\
	DAQ 		& Data Acquisition		 			& \\
	H			& Head								& $bar$\\
	MFM 		& Magnetic Flow Meter 				& \\
	MIMO		& Multiple Input Multiple Output	& \\
	ML			& MATLAB\textsuperscript{\textregistered{}} & \\
	NPSH		& Net Positive Suction Head 		& $bar$\\
	OL			& Open-loop							& \\
	OS			& Operating System					& \\
	$P$			& Power								& $W$\\
	PC			& Personal Computer					& \\
	SISO		& Single Input Single Output		& \\
	SL			& Simulink\textsuperscript{\textregistered{}} & \\
	SLRT		& Simulink\textsuperscript{\textregistered{}} Real-Time\texttrademark{}& \\

	$\Delta P$	& Pressure Difference				& $bar$\\
	$\eta$		& Efficiency						& $\%$\\
	$\rho$		& Density							& $\frac{kg}{m^3}$\\
	$\omega$	& Rotational speed					& $\frac{rad}{s}$\\
	\hline \hline
				& 									&	\\
				& \textbf{Subscripts}				&	\\
	\hline
	$els$		& Electric							&	\\
	$hyd$		& Hydraulic							&	\\
	$ref$		& Reference							&	\\
	$tot$		& Total								&	\\
	\hline \hline
				& 									&	\\
				& \textbf{Prescripts}				&	\\
	\hline
	$\Delta$	& Change							&	\\
	\hline \hline
\end{tabular*}
\todo[color=01introduction]{We have to ctrl+fuck-it for every acronym, 
abbreviation or nomenclature, just to make sure we actually use and explain it}


\chapter{Introduction}\label{ch:introduction}
The human body consists of 60\% water \cite{HumanWater}.
To remain healthy, 
it is recommended to consume between 2,7 l and 3,7 l of water every day \cite{DailyWater}.
This intake of water comes from both drinks and food.
The water is often added during growth, processing, or preparation of food.
For drinks to reach the consumer it has to be transported, 
in the form of bottles, tanks, or through pipelines.

The ability to precisely control liquids is also very important in many industries.
Examples range from the oil and gas sector \cite{OilFlow} over breweries \cite{BrewFlow},
dairy plants \cite{DairyFlow} to waste water treatment plants \cite{WastewaterFlow}.
In every commercial use,
it is important to keep cost low.
The lifetime cost of any system depends both on the initial cost,
but especially the running cost \cite{LifetimeCost}.
The running cost of controlling flow depends largely on the efficiency of the pumps used \cite{LifeCycleCostEfficiency}.

Prior work in the field shows, 
that efficiency can be improved by changing from a single pump to multiple pumps,
when the scheduling of those are optimally controled \cite{YangMultiPump2008}. 
The mentioned research only focuses on controlling pressure,
and it would therefore be interesting to investigate, 
whether this also is true for controlling flow.

\todo[color=01introduction]{I left 1-3 here for us to choose from since it should go well with the conclusion as well.}

1. The focus of this report therefore lays on controlling a constant flow,
inside a working range optimized for minimal energy consumption.
\\

2. The focus of this project is to develop a controller for a system of pumps in such a way,
that a constant flow is achieved with a minimal power consumption.
\\

3. The main focus is to understand, develop and implement modelling and control techniques.
We will compare different techniques for both modelling and control in this report
and conclude why specific techniques fit better to our purpose.
%
%
%For the water to come into the food,
%plants need to be watered regularly.
%For drinking water (or other liquids), they have to be transported.
%
%In any way, the flow of the water has to be controlled at some point in this chain,
%be it for irrigation or for bottling drinks.
%
%Even though 
%
%The focus of this project is to develop a controller for a system of pumps in such a way,
%that a constant flow is achieved with a minimal power consumption.
%
%The main focus is to understand, develop and implement modelling and control techniques.
%We will compare different techniques for both modelling and control in this report
%and conclude why specific techniques fit better to our purpose.
%
%
%Flow control is important in several situations throughout many industries.
%
%
%Controlled flow can be achieved relatively easily by controlling the shaft rotation speed $\omega$,
%but this can lead to extreme power consumption.
%\todo[inline,color=01introduction]{refer to a later chapter, something about affinity laws linear vs. cubed}
%
%According to the affinity laws%\cite{AffinityLaws}
%

\section*{Reading Guide}
In Chapter \ref{ch:physsetup} the physical setup gets explained,
giving a short introduction to all components relevant for this report.
Chapter \ref{ch:mathmodel} and \ref{ch:modValPerf}
focus on the mathematical modelling of those components and their interdependencies.
Focus is mainly put on the relations between Flow, Head and Efficiency according to a given Pump Speed.
Different modelling techniques are explained and evaluated,
alongside a very brief introduction to MATLAB\textsuperscript{\textregistered{}},
Simulink\textsuperscript{\textregistered{}} Real-Time\texttrademark{} and xPC Target,
the software used for this process.
Chapter \ref{ch:experiment} explains how the experiments were conducted
and what results were achieved.
Chapter \ref{ch:controldesign} covers how the model was used to design a controller
fulfilling our requirements stated in Chapter \ref{ch:probdesc},
the implementation of which,
again using
MATLAB\textsuperscript{\textregistered{}},
Simulink\textsuperscript{\textregistered{}} Real-Time\texttrademark{}
and xPC Target gets explained in Chapter \ref{ch:cimplement}. \todo[color=01introduction]{After Chapter 2, this sentence makes no sense to me?}
Chapters \ref{ch:discussion} and \ref{ch:conclusion} summarize the findings from this work,
propose possible future work and pose a conclusion to the work done.
