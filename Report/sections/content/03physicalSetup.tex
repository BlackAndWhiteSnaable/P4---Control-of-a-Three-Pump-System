\chapter{Physical Setup}\label{ch:physsetup}
\section{Centrifugal Pumps}

A pump is device used to move liquid through a piping system and to raise the pressure of the liquid. We will focus on explaining 
and describing only centrifugal pumps, since this is the type of pump present in our setup. 

\subsection{Principle}

In 1689, physicist Denis Papin invented the centrifugal pump. It is the most commonly used type of pump,
due to its simple construction, relative low cost, reliability and quiet operation.

The pump is built on a simple principle: Liquid is led to the impeller hub and by means of the centrifugal force
it is flung towards the periphery of the impeller. 

Figure \ref{fig:pump_sections} represents two cross sections of a centrifugal pump.
\begin{figure}[h]
    \centering
    \includegraphics[width=0.3\linewidth]{figures/pump_cross_section.PNG}
    \qquad
    \hfill
    \includegraphics[width=0.3\linewidth]{figures/pump_above_view.PNG}
    \caption{Centrifugal Pump}
    \label{fig:pump_sections}
\end{figure}
\newpage

The fluid is sucked into the impeller at the impeller eye and flows through the impeller channels formed by the 
blades between the shroud and hub.
The blades of the rotating impeller transfer energy to the fluid by increasing velocity and pressure.

The design of the impeller depends on the requirements for application, pressure and flow.
The impeller is the primary component determining the pump performance. 
Pumps variants are often created only by modifying the impeller.

\subsection{Affinity Laws}
Affinity laws are mathematical relationships that provide a way to estimate the changes in performance of a pump, 
as a result of a change in one of the basic pump variables.
In it's simplest form, the term law, means a principle that has been proven true for all cases.

Equations for a specific centrifugal pump to determine flow, head and power curves for different motor speeds $N$ \cite{Volk2014}.
\begin{align*}
	\frac{Q_1}{Q_2} = \left(\frac{N_1}{N_2}\right)   &&
	\frac{H_1}{H_2} = \left(\frac{N_1}{N_2}\right)^2 &&
	\frac{P_1}{P_2} = \left(\frac{N_1}{N_2}\right)^3	
\end{align*} 

Equations for a specific centrifugal pump to determine flow, head and power curves for different impeller sizes $D$ \cite{Volk2014}.

\begin{align*}
	\frac{Q_1}{Q_2} = \left(\frac{D_1}{D_2}\right)   &&
	\frac{H_1}{H_2} = \left(\frac{D_1}{D_2}\right)^2 &&
	\frac{P_1}{P_2} = \left(\frac{D_1}{D_2}\right)^3
\end{align*} 

\subsection{Performance Curves}
\subsubsection{Pump Head Curve}
A QH-curve or pump curve defines the head as a function of the flow. The flow is the rate of the fluid going through the 
pump. It is generally stated in cubic meter per hour $[m^{3}/h]$. Figure \ref{fig:pump_head_curve} represents a typical pump head curve.
\todo[color=04mathematicalModelling]{insert pump curve photo}
\subsubsection{Power Curve}
\subsubsection{Efficiency Curve}


\subsubsection{System Head Curve}
A system head curve is a graphical representation of the pump head that is required 
to move fluid through a piping system at various flow rates.
The system curve helps quantify the resistance in a system due to friction and 
elevation change over the range of flows.

\todo[color=04mathematicalModelling]{insert system curve photo}




\section{Pipes}
Pipes are a way of transporting liquids or gasses, inside a controllable environment.
They are used to interconnect the pumps and the tank and other peripherals.
One common analogy compares them to wires in electrical circuits.

Based on their diameters, material and shape,
they introduce resistance to the flow of the pumped medium.
Staying with the analogy to electrical circuits,
this can be compared to the cross sectional area and the specific resistance of a wire material.

\section{Valves}
A valve is a device used to regulate the flow of a gas or liquid through a piping system.
The valve built into our system is not used for regulatory purposes,
but only to simulate disturbances in the system.
Valves can be actuated by different means, such as air pressure, electric motors or rotary handles.

\section{Sensors}
Sensors were present on the setup. They were used to gather the data used throughout the report 
and closely monitor the system.

\subsection{Flow Meter}
\subsection{$\Delta$Pressure Sensor}
\subsection{Power Sensor}
\todo[color=03physicalSetup]{complete this list of subsections}
\todo[color=03physicalSetup]{maybe take out of ToC? (subsection*)}

\todo[color=03physicalSetup]{Do we want a section about xPC and Simulink Realtime?}
