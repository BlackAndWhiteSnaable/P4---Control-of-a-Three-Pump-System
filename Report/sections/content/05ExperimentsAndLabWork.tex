\chapter{Experiments and Lab Work}\label{ch:experiment}
Introduction goes here... 
 
\section{Performance test}\label{sec:performance_test} 
For gathering data about the pump system, a performance test was carried out.
\todo{Should we explain how a performance test is carried out?}

Instead of relying on performance curves provided by the pump manufacturer,
the pump will be run in situ, and the obtained data will show how the pump 
operates for the specific system.
 
During the test gauges for flow, pressure, speed, and power consumption were 
measured. Flow resistance can be varied by a choke valve, resulting in 
corresponding values of flow, differential pressure, and power consumption 
which has been measured in order to create the performance curves.
 
The actual code for the pump test application was done in Matlab. 
A model of the system was made using Simulink blocks. \todo{elaborate on this?}
Simulink Real-time and xPC Target was used to run the model in real-time 
on the pump system's dedicated PC. 
\todo{move somewhere else?} 

xPC Target allows you to add I/O blocks to your model, and then use the host 
PC and a C compiler to create executable code. The executable code is download 
from the host PC to the target PC running the xPC Target real-time kernel. 
After downloading the executable code, you can run and test your target 
application in real-time. \todo{nescesarry?}

% The physical system and what sensors/variables avaliable should already be explained 
- Experiment setup (What are we controlling and doing - explain the script)
- Data acquisition (other topic now?)
- Results
\todo{Do we need the datasheet for the pump? We must be using it for something?? (NPSHR?)}
 