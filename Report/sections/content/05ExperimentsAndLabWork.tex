\chapter{Experiments and Lab Work}\label{ch:experiment} 
\section{System test}\label{sec:system_test} 
A test is carried out in order to obtain information about how the specific 
system reacts under different load conditions. A single pump is run 
at different speeds, while flow resistance is varied by a choke valve, resulting 
in corresponding values of flow, pressure, and energy consumption measured by 
the sensors. The measured data is stored for further processing such as creating 
curves.

\section{Results}\label{sec:results}
Explain this better:\newline
- The pump speed is gradually turned up from 0 to 100\% in 10\% intervals. \newline
- For each pump speed, the valve openningvalve settling time, steps of 10 seconds\newline
- Select valve opening(backpressure) by for loop\newline

\missingfigure[figwidth=0.5\textwidth]{Show curves from early test runs}

\missingfigure[figwidth=0.5\textwidth]{Show curves from early test runs}

\section{Data gathering}\label{sec:data_gathering}
The script programmed for the system test was done in Matlab. 
A model of the system is made using Simulink blocks. 
Simulink Real-time and xPC Target was used to run the model in real-time on 
the pump systems dedicated PC. 
\todo[color=05ExperimentsAndLabWork]{Is this ok here? Do we want a seperate section for this anyway?}




% Explain how we carried out the performance test
xPC Target allows you to add input/output blocks to your model, and then use the host 
PC and a C compiler to create executable code. The executable code is download 
from the host PC to the target PC running the xPC Target real-time kernel. 
After downloading the executable code, you can run and test your target 
application in real-time. \todo[color=05ExperimentsAndLabWork]{is it necessary to describe this and data acquisition?}

\missingfigure[figwidth=0.5\textwidth]{Placeholder figure}

Remove this nonsense:
A performance curve is plotted to indicate the variation of pump differential head against volumetric flow (gpm) of a liquid at an indicated rotational speed or velocity, while consuming a specific quantity of horsepower (BHP). The performance curve is actually four curves relating with each other on a common graph. These four curves are:

 