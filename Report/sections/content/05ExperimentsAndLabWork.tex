\chapter{Experiments and Lab Work}\label{ch:experiment} 
\section{Performance test}\label{sec:performance_test} 
% Explain why it is important to carry out a performence test

\todo[color=05ExperimentsAndLabWork]{Should we explain how a performance test is carried out?}

A performance curve is plotted to indicate the variation of pump differential head against volumetric flow (gpm) of a liquid at an indicated rotational speed or velocity, while consuming a specific quantity of horsepower (BHP). The performance curve is actually four curves relating with each other on a common graph. These four curves are:

For gathering data about the pump, a performance test was carried out.

Instead of relying on performance curves provided by the pump manufacturer,
the pump is run in situ, and the obtained data will show how the pump operates for the specific system.

During the test, flow resistance is varied by a choke valve, resulting in 
corresponding values of flow, differential pressure, and power consumption 
which has been measured in order to create the performance curves.
\newline
\newline
The programming for the pump test was done in Matlab. 
A model of the system was made using Simulink blocks. \todo[color=05ExperimentsAndLabWork]{elaborate on this?}
Simulink Real-time and xPC Target was used to run the model in real-time 
on the pump systems dedicated PC. 
\todo[color=05ExperimentsAndLabWork]{move somewhere else?}

\missingfigure[figwidth=0.5\textwidth]{Placeholder figure}

% Explain how we carried out the performance test
xPC Target allows you to add input/output blocks to your model, and then use the host 
PC and a C compiler to create executable code. The executable code is download 
from the host PC to the target PC running the xPC Target real-time kernel. 
After downloading the executable code, you can run and test your target 
application in real-time. \todo[color=05ExperimentsAndLabWork]{is it necessary to describe this and data acquisition?}

\missingfigure[figwidth=0.5\textwidth]{Placeholder figure}

\textbf{Experiment setup}
%- pumpspeeds to iterate through: 0:10:100\n
- time intervals from valve settling time, steps of 10 seconds
- select valve opening(backpressure) by for loop


- Results
\todo[color=05ExperimentsAndLabWork]{Do we need the datasheet for the pump? We must be using it for something?? (NPSHR?)}
 