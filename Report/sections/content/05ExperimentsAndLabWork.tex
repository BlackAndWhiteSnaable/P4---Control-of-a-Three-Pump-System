\chapter{Experiments and Lab Work}\label{ch:experiment} 
\section{Performance test}\label{sec:performance_test} 
For gathering data about the pump system, a performance test was carried out.
\todo{Should we explain how a performance test is carried out?}
\todo{yes -Daniel wait, we do later, no?}

Instead of relying on performance curves provided by the pump manufacturer,
the pump will be run in situ \todo{"situ"? I think something went missing here}, and the obtained data will show how the pump 
operates for the specific system.
 
During the test flow resistance is varied by a choke valve, resulting in 
corresponding values of flow, differential pressure, and power consumption 
which has been measured in order to create the performance curves.
\newline
\newline
The programming for the pump test was done in Matlab. 
A model of the system was made using Simulink blocks. \todo{elaborate on this?}
Simulink Real-time and xPC Target was used to run the model in real-time 
on the pump systems dedicated PC. 
\todo{move somewhere else?} 

\missingfigure[figwidth=0.5\textwidth]{Placeholder figure}

xPC Target allows you to add I/O blocks to your model, and then use the host 
PC and a C compiler to create executable code. The executable code is download 
from the host PC to the target PC running the xPC Target real-time kernel. 
After downloading the executable code, you can run and test your target 
application in real-time. \todo{is it nescesarry to describe this and data acquisition?}

\missingfigure[figwidth=0.5\textwidth]{Placeholder figure}

\textbf{Experiment setup}
- pumpspeeds to iterate through: 0:10:100\n
- time intervals from valve settling time, steps of 10 seconds
- select valve opening(backpressure) by for loop


- Results
\todo{Do we need the datasheet for the pump? We must be using it for something?? (NPSHR?)}
 