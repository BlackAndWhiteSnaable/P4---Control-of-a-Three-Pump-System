\chapter{Experiments and Lab Work}\label{ch:experiment} 
\section{System test}\label{sec:system_test} 
A test is carried out in order to obtain information about how the specific 
system reacts under different load conditions. A single pump is run at 
different speeds, while flow resistance is varied by a choke valve, resulting 
in measuring corresponding values of flow and pressure. Curves are created 
from the measured data.

Remove this:
A performance curve is plotted to indicate the variation of pump differential head against volumetric flow (gpm) of a liquid at an indicated rotational speed or velocity, while consuming a specific quantity of horsepower (BHP). The performance curve is actually four curves relating with each other on a common graph. These four curves are:

\section{Data gathering}\label{sec:data_gathering} 
\textbf{Experiment setup}
%- pumpspeeds to iterate through: 0:10:100\n
- time intervals from valve settling time, steps of 10 seconds
- select valve opening(backpressure) by for loop


- Results
\todo[color=05ExperimentsAndLabWork]{Do we need the datasheet for the pump? We must be using it for something?? (NPSHR?)}

The programming for the pump test was done in Matlab. 
A model of the system was made using Simulink blocks. \todo[color=05ExperimentsAndLabWork]{elaborate on this?}
Simulink Real-time and xPC Target was used to run the model in real-time 
on the pump systems dedicated PC. 
\todo[color=05ExperimentsAndLabWork]{move somewhere else?}

\missingfigure[figwidth=0.5\textwidth]{Placeholder figure}

% Explain how we carried out the performance test
xPC Target allows you to add input/output blocks to your model, and then use the host 
PC and a C compiler to create executable code. The executable code is download 
from the host PC to the target PC running the xPC Target real-time kernel. 
After downloading the executable code, you can run and test your target 
application in real-time. \todo[color=05ExperimentsAndLabWork]{is it necessary to describe this and data acquisition?}

\missingfigure[figwidth=0.5\textwidth]{Placeholder figure}


 